\chapter{The Physical Layer}
\label{cha:physicallayer}

TODO which modules? C++ interface

\begin{itemize}
  \item ongoing transmissions
  \item recent successful receptions
  \item recent obstacle intersections and surface normal vectors
\end{itemize}

\begin{verbatim}
TODO: 
 - exploit multiple CPUs and the highly parallel GPU to increase performance
 - provide performance vs. accuracy tradeoff configuration options
   (e.g. range filter, radio mode filter, listening mode filter, MAC address filter)
 - support different level of details (see details below)
 - support different transmitters and receivers (scale from flat to layered models)
 - support different radio signal models (scale from range based to accurate emulation models)
 - support different propagation models (scale from immediate to accurate models)
 - support different attenuation models (scale from free space to trace based models)
 - support different antenna models (scale from isotropic to directional models)
 - support different power consumption models (scale from mode based to signal based models)
 - provide concurrent transmitter and receiver mode (transceiver mode)
 - provide burst mode (back to back) transmissions
 - provide synchronization/preamble detection
 - provide capture during reception (switching to another transmission)
 - provide finite time radio mode switching

TODO: scalar vs dimensional
TODO: flat vs layered
TODO: Generic, IEEE 802.11, IEEE 802.15.4
TODO: acoustic underwater example
TODO: wireless vs. wired medium

\end{verbatim}


%%% Local Variables:
%%% mode: latex
%%% TeX-master: "usman"
%%% End:

