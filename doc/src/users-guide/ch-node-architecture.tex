\chapter{Node Architecture}
\label{cha:node-architecture}


\section{Overview}

This chapter describes the architecture of INET host and router models.

Hosts and routers in the INET Framework are OMNeT++ compound modules that
are composed of the following ingredients:

\begin{itemize}

\item \tbf{Interface Table (\nedtype{InterfaceTable})}. This module
contains the table of network interfaces (eth0, wlan0, etc) in the host.
Interfaces are registered dynamically during the initialization phase by
modules that represent network interface cards (NICs). Other modules access
interface information via a C++ class interface.

\item \tbf{Routing Table (\nedtype{IPv4RoutingTable})}. This module contains
the IPv4 routing table. It is also accessed from other modules via a C++ interface.
The interface contains member functions for adding, removing, enumerating
and looking up routes, and finding the best matching route for a given
destination IPv4 address. The IPv4 module calls the latter function for
routing packets, and routing protocols such as OSPF or BGP call the route
manipulation methods to add and manage discovered routes. For IPv6
simulations, \nedtype{IPv4RoutingTable} is replaced (or co-exists) with
a \nedtype{IPv6RoutingTable} module, possibly with a \nedtype{BindingCache}
module as well.

\ifdraft TODO
; and for Mobile IPv6 simulations (xMIPv6 project [TODO])
\fi

\ifdraft TODO
\item \tbf{Notification Board (\nedtype{NotificationBoard})}. This module
makes it possible for several modules to communicate in a publish-subscribe
manner. Notifications include change notifications (``routing table
changed'') and state changes (``radio became idle'').
\fi

\item \tbf{Mobility module}. In simulations involving node mobility, this
module is responsible for moving around the node in the simulated
``playground.'' A mobility module is also needed for wireless simulations
even if the node is stationary, because the mobility module stores the
node's location, needed to compute wireless transmissions. Different
mobility models (Random Walk, etc.) are supported via different module
types, and many host models define their mobility submodules with
parametric type so that the mobility model can be changed in the
configuration (\ttt{"mobility: <mobilityType> like IMobility"}).

\item \tbf{NICs}. Network interfaces are usually compound modules
themselves, composed of a queue and a MAC module (and in the case of
wireless NICs, a radio module or modules). Examples are
\nedtype{PPPInterface}, \nedtype{EthernetInterface}, and WLAN interfaces
such as \nedtype{Ieee80211NicSTA}. The queue submodule stores packets
waiting for transmission, and it is usually defined as having parametric
type as it has multiple implementations to accommodate different needs
(\nedtype{DropTailQueue}, \nedtype{REDQueue}, \nedtype{DropTailQoSQueue},
etc.) Most MACs also have an internal queue to allow operation without an
external queue module, the advantage being smaller overhead. The NIC's
entry in the host's \nedtype{InterfaceTable} is usually registered by the
MAC module at the beginning of the simulation.

\item \tbf{Network layer}. Modules that represent protocols of the network
layer are usually grouped into a compound module: \nedtype{IPv4NetworkLayer}
for IPv4, and \nedtype{IPv6NetworkLayer} for IPv6. \nedtype{IPv4NetworkLayer}
contains the modules \nedtype{IPv4}, \nedtype{ARP}, \nedtype{ICMP} and
\nedtype{ErrorHandling}. The \nedtype{IPv4} module performs IP
encapsulation/decapsulation and routing of datagrams; for the latter it
accesses the C++ function call interface of the \nedtype{IPv4RoutingTable}.
Packet forwarding can be turned on/off via a module parameter. The
\nedtype{ARP} module is put into the path of packets leaving the network
layer towards the NICs, and performs address resolution for interfaces that
need it (e.g. Ethernet). \nedtype{ICMP} deals with sending and receiving
ICMP packets. The \nedtype{ErrorHandling} module receives and logs ICMP
error replies. The IPv6 network layer, \nedtype{IPv6NetworkLayer} contains the
modules \nedtype{IPv6}, \nedtype{ICMPv6}, \nedtype{IPv6NeighbourDiscovery}
and \nedtype{IPv6ErrorHandling}. For Mobile IPv6 simulations 

\ifdraft TODO
(xMIPv6 project [TODO]),
\fi

\nedtype{IPv6NetworkLayer} is extended with further modules.

\item \tbf{Transport layer protocols}. Transport protocols are represented
by modules connected to the network layer; currently TCP, UDP and SCTP are
supported. TCP has several implementations: \nedtype{TCP} is the OMNeT++
native implementation; the \nedtype{TCP\_lwip} module wraps the lwIP TCP
stack \ifdraft [TODO] \fi; and the \nedtype{TCP\_NSC} module wraps the Network
Simulation Cradle library \ifdraft [TODO] \fi. For this reason, the \ttt{tcp} 
submodule is usually defined with a parametric submodule type (\ttt{"tcp:
<tcpType> like ITCP"}). UDP and SCTP are implemented by the \nedtype{UDP} and
\nedtype{SCTP} modules, respectively.

\item \sloppypar \tbf{Applications}. Application modules typically connect to TCP
and/or UDP, and model the user behavior as well as the application program
(e.g. browser) and application layer protocol (e.g. HTTP). For convenience,
\nedtype{StandardHost} supports any number of UDP, TCP and SCTP
applications, their types being parametric (\ttt{"tcpApp[numTcpApps]:
<tcpAppType> like TCPApp; udpApp[numUdpApps]: <udpAppType> like
UDPApp; ..."}). This way the user can configure applications entirely from
\ttt{omnetpp.ini}, and does not need to write a new NED file every time
different applications are needed in a host model. Application modules
are typically not present in router models.

\item \tbf{Routing protocols}. Router models typically contain modules that
implement routing protocols such as OSPF or BGP. These modules are
connected to the TCP and/or the UDP module, and manipulate routes in the
\nedtype{IPv4RoutingTable} module via C++ function calls.

\item \tbf{MPLS modules}. Additional modules are needed for MPLS
simulations. The \nedtype{MPLS} module is placed between the network layer
and NICs, and implements label switching. \nedtype{MPLS} requires a
\nedtype{LIB} module (Label Information Base) to be present in the router
which it accesses via C++ function calls. MPLS control protocol
implementations (e.g. the \nedtype{RSVP} module) manage the information in
\nedtype{LIB} via C++ calls.

\item \tbf{Relay unit}. Ethernet (and possibly other) switch models may
contain a relay unit, which switches frames among Ethernet (and other
IEEE 802) NICs. Concrete relay unit types include \nedtype{MACRelayUnit}
and \nedtype{Ieee8021dRelay}, which differ in their performance models.

\ifdraft TODO
\item \tbf{Power consumption}. INET extensions uses for wireless sensor
networks (WSNs) may add a battery module to the node model. The battery
module would keep track of energy consumption. A battery module is provided
e.g. by the INETMANET project.
\fi

\end{itemize}

The \nedtype{StandardHost} and \nedtype{Router} predefined NED types are
only one possible example of host/router models, and they do not contain
all the above components; for specific simulations it is a perfectly
valid approach to create custom node models.

Most modules are optional, i.e. can be left out of hosts or other node
models when not needed in the given scenario. For example, switch models do
not need a network layer, a routing table or interface table. Some NICs (e.g.
\nedtype{EtherMAC}) can be used without and interface table and queue models as
well.

\ifdraft TODO
The notification board (\nedtype{NotificationBoard}) and the interface
table (\nedtype{InterfaceTable}) will be described later in this chapter.
Other modules are covered in later chapters, i.e. \nedtype{IPv4RoutingTable}
in the IPv4 chapter.
\fi





%%% Local Variables:
%%% mode: latex
%%% TeX-master: "usman"
%%% End:

